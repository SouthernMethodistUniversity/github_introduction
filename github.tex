\documentclass[aspectratio=169]{beamer}
   \usetheme{metropolis}
   \setbeamertemplate{blocks}[rounded][shadow=false]
\usepackage{url}
\usepackage{hyperref}
\usepackage{booktabs}
\usepackage{tabularx}
\usepackage{dcolumn}
   \newcolumntype{d}[1]{D{.}{.}{#1}}
\usepackage{graphicx}
\usepackage[justification=centering]{caption}
\usepackage{adjustbox}
\usepackage{color}
\usepackage{textpos}
\usepackage{etoolbox}
\usepackage[cache=false]{minted}
\usepackage{multimedia}

\makeatletter
\patchcmd{\beamer@sectionintoc}{\vskip1.5em}{\vskip0.5em}{}{}
\makeatother

\definecolor{smured}{rgb}{0.797,0,0.027}
\definecolor{smublue}{RGB}{48,64,116}
\definecolor{dkgreen}{rgb}{0,0.6,0}
\definecolor{gray}{rgb}{0.5,0.5,0.5}
\definecolor{mauve}{rgb}{0.58,0,0.82}
\definecolor{text_gray}{RGB}{46,58,62}

\setbeamercolor{progress bar}{fg=smured,bg=smublue}
\setbeamercolor{title separator}{fg=smublue}
\setbeamercolor{frametitle}{bg=smublue}

\metroset{
  numbering=fraction
}

\hypersetup{
  colorlinks=true,
  allcolors=text_gray,
  urlcolor=smured,
}

\addtobeamertemplate{frametitle}{}{
\begin{textblock*}{1cm}(\textwidth,-1.155cm)
\includegraphics[width=1cm]{figures/smu_logo.pdf}
\end{textblock*}}

\setminted{breaklines,linenos,fontsize=\scriptsize}
\setmintedinline{fontsize=auto}

\title{GitHub Introduction}
\author{Robert Kalescky\\ HPC Applications Scientist}
\institute{
Research and Data Sciences Services\\
Office of Information Technology\\
Center for Research Computing\\
Southern Methodist University}
\date{August 26, 2021}

\begin{document}

{
\setbeamertemplate{navigation symbols}{}
\begin{frame}[plain,noframenumbering]
    \makebox[\linewidth]{\includegraphics[width=\paperwidth]{figures/DataIsntScary.jpg}}
\end{frame}
}

\begin{frame}
\titlepage
\end{frame}

\begin{frame}{Outline}
\footnotesize
\tableofcontents[hideallsubsections]
\end{frame}

\section{Research Support}

\begin{frame}{Research and Data Science Services}
\begin{itemize}
  \item Provide research computing support, consultations, and collaborations
  \item Data Science - Dr. Eric Godat
  \item High-Performance Computing - Dr. Robert Kalescky \& Dr. John LaGrone
  \item Custom Devices (IOT, wearables, etc.) - Guillermo Vasquez
\end{itemize}
\end{frame}

\begin{frame}{Center for Research Computing (CRC)}
\begin{itemize}
  \item Maintains our primary shared resource for research computing, ManeFrame II (M2), in collaboration with OIT
  \item Provides research computing tools, support, and training to all faculty, staff, and students using research computing resources
  \item \url{www.smu.edu/crc} has documentation and news
  \item \href{mailto:help@smu.edu}{help@smu.edu} or \href{mailto:rkalescky@smu.edu}{rkalescky@smu.edu} for help
  \item Request an account at \url{www.smu.edu/crc}
\end{itemize}
\end{frame}



\section{Version control systems}

\begin{frame}{Version control systems}

Version Control (aka \emph{revision control} or \emph{source control})
lets you track the history of your files over time. Why do you care? So
when you mess up you can easily get back to a previous version that
worked.

You've probably invented your own simple version control system in the
past without realizing it. Do you have an directories with files like
this?

\end{frame}

\begin{frame}{Version control systems}

\begin{itemize}
\item
  \mintinline{sh}{my_function.c}
\item
  \mintinline{sh}{my_function2.c}
\item
  \mintinline{sh}{my_function3.c}
\item
  \mintinline{sh}{my_function4.c}
\item
  \mintinline{sh}{my_function_old.c}
\item
  \mintinline{sh}{my_function_older.c}
\item
  \mintinline{sh}{my_function_even_older.c}
\end{itemize}

\end{frame}

\begin{frame}{Version control systems}

It's why we use "Save As"; you want to save the new file without writing
over the old one. It's a common problem, and solutions are usually like
this:

\begin{itemize}
\item
  Make a \emph{single backup copy} (e.g.~Document.old.txt).
\item
  If we're clever, we add a \emph{version number} or \emph{date}: e.g.
  Document\_V1.txt, DocumentMarch2012.txt.
\item
  We may even use a \emph{shared folder} so other people can see and
  edit files without sending them by email. Hopefully they rename the
  file after they save it.
\end{itemize}

\end{frame}

\begin{frame}{Why use a VCS?}

\begin{itemize}
\item Our shared folder/naming system is fine for class projects or one-time
papers, but is exceptionally bad for software projects.
\item Imagine
that the Windows source code sits in a shared folder named something
like "Windows10-Latest-New", for anyone to edit? Or that every programmer
just works on different files in the same folder?
\item For projects that are large, fast-changing, or have multiple authors, a
Version Control System (VCS) is critical.
\item Think of a VCS as a "file
database", that helps to track changes and avoid general chaos.
\end{itemize}

\end{frame}

\begin{frame}{Why use a VCS?}

\begin{description}
\item[Backup and Restore] files are saved as they are edited, and
  you can jump to any moment in time. Need that file as it was on March
  8? No problem.
\item[Synchronization] Allows people to share files and stay up
  to date with the latest version.
\item[Short-term undo] Did you try to ``fix'' a file and just mess
  it up? Throw away your changes and go back to the last "correct"
  version in the database.
\item[Long-term undo] Sometimes we mess up bad. Suppose you made
  a change a year ago, and it had a bug that you never caught until now.
  Jump back to the old version, and see what change was made that day.
  Maybe you can fix that one bug and not have to undo your work for the
  whole year?
\end{description}
\end{frame}

\begin{frame}{Why use a VCS?}

\begin{description}
\item[Track Changes] As files are updated, you can leave messages
  explaining why the change happened (these are stored in the VCS, not
  the file). This makes it easy to see how a file is evolving over time,
  and why it was changed.
\item[Track Ownership] A VCS tags every change with the name of
  the person who made it, which can be hepful for laying blame \emph{or}
  giving credit.
\item[Sandboxing] (i.e.~insurance against yourself) -\/- Plan to make
  a big change? You can make temporary changes in an isolated area, test
  and work out the kinks before "checking in" your set of changes.
\item[Branching and merging] A larger sandbox. You can branch a
  copy of your code into a separate area and modify it in isolation
  (tracking changes separately). Later, you can merge your work back
  into the common area.
\end{description}
\end{frame}

\section{Definitions}

\begin{frame}{General definitions}

\begin{description}
\item[Repository (repo)] The database storing the files.
\item[Server] The computer storing the repo.
\item[Client] The computer connecting to the repo.
\item[Working Copy] Your local directory of files,
  where you make changes.
\item[Trunk/Main] The primary location for code in the repo.
  Think of code as a family tree --- the trunk is the main line.
\end{description}

\end{frame}

\section{Usage}

\begin{frame}{Basic actions}

\begin{description}
\item[Add] Put a file into the repo for the first time,
  i.e.~begin tracking it with Version Control.
\item[Revision] What version a file is on (v1, v2, v3, etc.).
\item[Head/Tip] The latest revision in the repo.
\item[Check Out] Download a file from the repo.
\item[Check In] Upload a file to the repository (if it has
  changed). The file gets a new revision number, and people can ``check
  out'' the latest one.
  \end{description}
  
  \end{frame}

\begin{frame}{Basic actions}

\begin{description}
\item[Checkin Message] A short message describing what was
  changed.
\item[Changelog/History] A list of changes made to a file since
  it was created.
\item[Update/Sync] Synchronize your files with the latest from
  the repository. This lets you grab the latest revisions of all files.
\item[Revert] Throw away your local changes and reload the latest
  version from the repository.
\end{description}

\end{frame}

\begin{frame}{More advanced actions}

\begin{description}
\item[Branch] Create a separate copy of a file/folder for private
  use (bug fixing, testing, etc). Branch is both a verb ("branch the
  code") and a noun ("Which branch is it in?").
\item[Diff/Change/Delta] Finding the differences between two
  files. Useful for seeing what changed between revisions.
\item[Merge/Patch] Apply the changes from one file to another, to
  bring it up-to-date. For example, you can merge features from one
  branch into another.
\item[Conflict] When pending changes to a file contradict each
  other (both changes cannot be applied automatically).
\end{description}
\end{frame}

\begin{frame}{More advanced actions}

\begin{description}
\item[Resolve] Fixing the changes that contradict each other and
  checking in the final version.
\item[Locking] Taking control of a file so nobody else can edit
  it until you unlock it. Some version control systems use this to avoid
  conflicts.
\item[Breaking the lock] Forcibly unlocking a file so you can
  edit it. It may be needed if someone locks a file and goes on
  vacation.
\item[Check out for edit] Checking out an "editable" version of a
  file. Some VCSes have editable files by default, others require an
  explicit command.
\end{description}

\end{frame}

\begin{frame}{Typical Usage}

\begin{itemize}
\item
  Alice adds a file (ShoppingList.txt) to the repository.
\item
  Alice checks out the file, makes a change (puts "milk" on the list),
  and checks it back in with a checkin message ("Added delicious
  beverage.").
\item
  The next morning, Bob updates his local working set and sees the
  latest revision of ShoppingList.txt, which contains "milk".
\item
  Bob adds "donuts" to the list, while Alice also adds "eggs" to the
  list.
\item
  Bob checks the list in, with a checking message
  \href{https://www.youtube.com/watch?v=8-4P1WPE-Qg}{"Mmmmm, donuts"}.
\item
  Alice updates her copy of the list before checking it in, and notices
  that there is a conflict. Realizing that the order of items doesn't
  matter, she merges the changes by putting both "donuts" and "eggs" on
  the list, and checks in the final version.
\end{itemize}

\end{frame}

\section{Git}

\begin{frame}{Git}

Originally released in 2005 (by
\href{https://en.wikipedia.org/wiki/Linus_Torvalds}{Linus Torvalds}
himself!), \href{https://en.wikipedia.org/wiki/Git_(software)}{Git} was
one of the first version control systems that followed a
\emph{distributed revision control} model (DRCS), in which it is no
longer required to have a single server that all clients connect with.
Instead, DRCS follows a peer-to-peer approach. in which each peer's
working copy of the codebase is a fully-functional repository. These
work by exchanging patches (sets of changes) between peers, resulting in
some
\href{https://en.wikipedia.org/wiki/Distributed_revision_control\#Distributed_vs._centralized}{key
benefits over previous centralized systems}

The
\href{https://confluence.atlassian.com/display/STASH/Basic+Git+commands}{commands}
used for interacting with Git are nearly identical to those for SVN,
with a few additions/exceptions:

\end{frame}

\begin{frame}{}

\begin{itemize}
\item
  \texttt{git\ clone} -\/- this is the primary mechanism for retrieving
  a local copy of a Git repository. Unlike the CVS and SVN
  \texttt{checkout} commands, the result is a full repository that may
  act as a server for other client repositories.
\item
  \texttt{git\ pull} -\/- this fetches and merges changes on the remote
  server to your working repository.
\item
  \texttt{git\ push} -\/- the opposite of \texttt{pull}, this sends all
  changes in your local repository to a remote repository.
\end{itemize}

\end{frame}

\begin{frame}{}

However, \emph{unlike SVN}, Git does not allow you to use the shortcut
names for standard commands; for example \texttt{git\ ci} is an illegal
command, but \texttt{git\ commit} is allowed.

While distributed version control systems no longer require a main
server, it is often useful to have a centralized, "agreed-upon" main
repository that all users can access. As with subversion, there are a
number of web sites that will host open-source Git-based software
projects free of charge, including:

\end{frame}

\begin{frame}{}

\begin{itemize}
\item
  \href{https://bitbucket.org/dashboard/overview}{Bitbucket}
\item
  \href{https://github.com/}{GitHub}
\item
  \href{https://gitorious.org/}{Gitorious}
\item
  \href{http://www.cloudforge.com/}{CloudForge}
\item
  \href{http://projectlocker.com/}{ProjectLocker}
\item
  \href{http://offers.assembla.com/free-git-hosting/}{Assembla}
\end{itemize}

\end{frame}

\begin{frame}{Git Resources}

\begin{itemize}
\item
  \href{http://git-scm.com/}{Main Git site}
\item
  \href{http://www.atlassian.com/git/tutorial}{Git tutorials}
\item
  \href{http://git-scm.com/book}{Git book chapters}
\end{itemize}

\end{frame}

\section{GitHub}

\begin{frame}{GitHub Instances}

SMU faculty, staff, and students have access to two distinct instances
of GitHub.

\begin{itemize}
\item
  \href{https://www.github.com}{github.com}

  \begin{itemize}
    \item
    Sign-up for free at \href{https://www.github.com}{github.com}
  \item
    Sign-up for GitHub Education Benefits at
    \href{https://education.github.com}{education.github.com}, which
    provides access to most of the pay-only benefits
    \end{itemize}
\item
    \href{https://github.smu.edu}{github.smu.edu}
    \begin{itemize}
  \item
    Log into \href{https://github.smu.edu}{github.smu.edu} using your
    SMU credentials
  \item
    Has all premium features
  \item
    Only private repositories
  \end{itemize}
\end{itemize}

\end{frame}

\begin{frame}{Repository Walkthrough}

\begin{enumerate}
\item
  Creating repositories
\item
  Branching
\item
  Merging
\item
  Pushing and pulling
\item
  Pull requests
\item
  Forking
\end{enumerate}

\end{frame}

\begin{frame}{Useful Repository Management Tips}

\begin{enumerate}
\item
  SSH keys
\item
  Branch restrictions
\item
  Project management
\item
  Actions
\end{enumerate}

\end{frame}

\end{document}
